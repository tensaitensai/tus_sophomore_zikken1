%4619055 課題1 
\documentclass[12pt]{jarticle}
\usepackage{TUSIreport}
\usepackage{otf}
\usepackage{graphicx}
\usepackage{amsmath}
\usepackage{amssymb}
\usepackage{hhline}
\usepackage{fancybox,ascmac}
\usepackage{url}
\usepackage{listings,jlisting}
\lstset{
    frame={tblr},
    basicstyle={\footnotesize},
    tabsize={4},
}
%%%%%%%%%%%%%%%%%%
\begin{document}
%%%%%%%%%%%%%%%%%%%%%%%%%%%%%%%%%%%%%%%%%%%%%%%%%%%%%%%%
% 表紙を出力する場合は,\提出者と\共同実験者をいれる
% \提出者{科目名}{課題名}{提出年}{提出月}{提出日}{学籍番号}{氏名}
% \共同実験者{一人目}{二人目}{..}{..}{..}{..}{..}{八人目}
%%%%%%%%%%%%%%%%%%%%%%%%%%%%%%%%%%%%%%%%%%%%%%%%%%%%%%%
\提出者{情報工学実験1}{課題1 テキスト処理}
{2020}{7}{20}{4619055}{辰川力駆}

\共同実験者{}{}{}{}{}{}{}{}
\追加実験者{}{}
\表紙出力

\section{実験内容}
Unix系OSのコマンドラインインターフェース上でテキストデータを加工する方法を学ぶ。
\subsection{レポート課題1}
\begin{shadebox}
    \LaTeX コード(sample.tex)をhtmlに変換するシェルスクリプトを作成せよ。
    ファイル名は``tex2html\_4619055.sh"とする。以下に本課題の注意事項を示す。
    \begin{itemize}
        \item 「情報工学実験1」という文字列は、$<$h2$>$...$<$/h2$>$で囲むこと。
        \item itemize環境は、htmlの$<$ul$>$環境に変換すること。
        \item enumerate環境、htmlの$<$ol$>$環境に変換すること。
        \item \LaTeX のコメント文は、htmlのコメント文に変換すること。削除してはならない。
        \item 作成するhtmlの文字コードはutf-8とすること。
    \end{itemize}
\end{shadebox}

\subsection{レポート課題2}
\begin{shadebox}
    課題1で作成したhtmlファイル中のコメント文$<$!$--$ End of ol $--$$>$の
    位置に新たにグループを追加するシェルスクリプトを作成せよ。
    ファイル名は``update\_html\_4619055.sh"とする。
    シェルスクリプトの実行時の引数を用いて、
    グループ名や学籍番号の範囲を指定せよ。
    具体的には、第1引数にグループ名を、第2引数と第3引数に学籍番号の範囲を記述する。
\end{shadebox}

\subsection{レポート課題3}
\begin{shadebox}
    課題2でhtmlファイルを更新する前に、バックアップを取るように``update\_html\_4619055.sh"を変更しなさい。
    変更後のシェルスクリプトのファイル名は、``update\_html\_ver2\_4619055.sh"とすること。
    また、``sample.html"のバックアップファイル名は``sample.html.更新日時"とする。
    ここで、``update\_html\_4619055.sh"が実行された時刻である。
\end{shadebox}

\subsection{レポート課題4}
\begin{shadebox}
    第1引数に``undo"という文字列を指定した場合に、更新時刻が最も新しいバックアップファイル
    で``sample.html"で上書きされ、元の状態を復元できるように``update\_html\_ver2\_4619055.sh"
    を変更しなさい。変更後のファイル名は、``update\_html\_ver3\_4619055.sh"とする。
\end{shadebox}

\clearpage

\section{実験結果}
\subsection{レポート課題1}
課題1の結果を下記に示した。正常に動いている。
\begin{lstlisting}
    $ bash tex2html_4619055.sh
    $ cat sample.html
    <!DOCTYPE html>
    <meta charset="utf-8">
    
    <body>
    
    <h2> 情報工学実験1</h2>
    
    TeX の基本操作と簡単なレポート作成の実習を行う.
    
    <ul style="list-style-type: circle">
        <li> 日時: 平成xx年xx月xx日 xx:xx -- xx:xx</li>
        <li> 場所: 東京理科大学葛飾キャンパス講義棟xxx教室</li>
    </ul>
    
    なお,グループ編成は<u>以下の通りとする</u>.
    
    <ol type="a">
        <li> G1 (学籍番号001 -- 020)</li>
        <li> G2 (学籍番号021 -- 040)</li>
        <li> G3 (学籍番号041 -- 060)</li>
        <li> G4 (学籍番号061 -- 080)</li>
     <!-- End of ol -->
    </ol>
    
    </body>
\end{lstlisting}
\subsection{レポート課題2}
課題2の結果を下記に示した。エラーハンドリングも含めて正常に動いている。
\begin{lstlisting}
    $ cat sample.html | tail -10

    <ol type="a">
        <li> G1 (学籍番号001 -- 020)</li>
        <li> G2 (学籍番号021 -- 040)</li>
        <li> G3 (学籍番号041 -- 060)</li>
        <li> G4 (学籍番号061 -- 080)</li>
     <!-- End of ol -->
    </ol>

    </body>
    $ bash update_html_4619055.sh
    エラー:グループ名を第1引数に,学籍番号の開始番号を第2引数に,
    終了番号を第3引数に指定してください.
    $ bash update_html_4619055.sh G5
    エラー:引数が足りません.
    $ bash update_html_4619055.sh G5 G5 G5 G5
    エラー:引数が多すぎます.
    $ bash update_html_4619055.sh G5 81 100
    $ bash update_html_4619055.sh G6 101 120
    $ cat sample.html | tail -12

    <ol type="a">
        <li> G1 (学籍番号001 -- 020)</li>
        <li> G2 (学籍番号021 -- 040)</li>
        <li> G3 (学籍番号041 -- 060)</li>
        <li> G4 (学籍番号061 -- 080)</li>
        <li> G5 (学籍番号081 -- 100)</li>
        <li> G6 (学籍番号101 -- 120)</li>
     <!-- End of ol -->
    </ol>

    </body>
\end{lstlisting}
\subsection{レポート課題3}
課題3の結果を下記に示した。バックアップが取れている。
\begin{lstlisting}
    $ ls
    sample.html  sample.tex  tex2html_4619055.sh  update_html_4619055.sh  
    update_html_ver2_4619055.sh
    $ bash update_html_ver2_4619055.sh G7 121 140
    $ ls 
    sample.html  sample.html.20200723060707  sample.tex  tex2html_4619055.sh  
    update_html_4619055.sh  update_html_ver2_4619055.sh
\end{lstlisting}
\subsection{レポート課題4}
課題4の結果を下記に示した。第1引数にundoを与えると最新のバックアップに戻っている。
\begin{lstlisting}
    $ cat sample.html | tail -10

    <ol type="a">
        <li> G1 (学籍番号001 -- 020)</li>
        <li> G2 (学籍番号021 -- 040)</li>
        <li> G3 (学籍番号041 -- 060)</li>
        <li> G4 (学籍番号061 -- 080)</li>
     <!-- End of ol -->
    </ol>

    </body>
    $ bash update_html_ver3_4619055.sh G5 81 100
    $ bash update_html_ver3_4619055.sh G6 101 120
    $ bash update_html_ver3_4619055.sh G7 121 140
    $ ls 
    sample.html  sample.html.20200723061544  sample.html.20200723061551  
    sample.html.20200723062320  sample.tex  tex2html_4619055.sh  
    update_html_4619055.sh  update_html_ver2_4619055.sh  
    update_html_ver3_4619055.sh
    $ bash update_html_ver3_4619055.sh undo
    Undo!
    $ diff -u sample.html sample.html.20200723062320
    $ diff -u sample.html sample.html.20200723061551
    --- sample.html	2020-07-23 06:24:16.706558000 +0900
    +++ sample.html.20200723061551	2020-07-23 06:15:51.742743000 +0900
    @@ -20,8 +20,7 @@
         <li> G3 (学籍番号041 -- 060)</li>
         <li> G4 (学籍番号061 -- 080)</li>
         <li> G5 (学籍番号081 -- 100)</li>
    -    <li> G6 (学籍番号101 -- 120)</li>
    - <!-- End of ol -->
    + <!-- End of ol -->
     </ol>
    
     </body>
\end{lstlisting}

\section{考察}
\subsection{レポート課題1}
tex2html\_4619055.shを作る上で用いたものは、主にsedコマンドである。
一つ一つに対しsedコマンドで書き換えて実行しようと考えた。
また、最後には無駄な行を省くためにいらない空白行を削除して、最後にsample.htmlに出力した。
\subsection{レポート課題2}
update\_html\_4619055.shを作る上で考えたのは、ifのエラー分岐である。
\$\#で引数の個数がわかるので、それが3以外のとき、エラー出力するように考えた。

また、出力するときは、sedを用いて、
もともと$<$!$--$ End of ol $--$$>$があった場所を、指定された文字を出力して改行した上で
        $<$!$--$ End of ol $--$$>$を出力することで実装している。

\subsection{レポート課題3}
update\_html\_ver2\_4619055.shは、update\_html\_4619055.shを改良したものであるが、
追加したのは一行だけである。具体的には、実行される前の段階で
\begin{lstlisting}
    cp sample.html sample.html.`date +"%Y%m%d%H%M%S"`
\end{lstlisting}
を追加することで、実装している。
dateコマンドで現在の日付や時刻を取得してそれをコピー先の名前に追加した。
\subsection{レポート課題4}
update\_html\_ver3\_4619055.shは、update\_html\_ver2\_4619055.shを改良したものである。
第1引数にundoという文字が入っていたならば、戻す作業を行うようにif文を加えた。
その後は、lsコマンドで``sample.html.日付"となっているものだけ取得し、
sortコマンドの-rオプションで降順に並べ替え、
最後にheadコマンドで一行だけ取得した。

\section{結論}
本実験を通して、シェルの基本機能及びUnix系OSの基本的なコマンドについて学ぶことができた。
また、コマンドラインインターフェース上で
sedとawkを用いたテキストデータの処理技法についても学ぶことができた。

\clearpage
\appendix
%%%%%%%%%%%%%%%%%%%%%%%%%%%%%%%%%%%%%%%%%%%%%%%%%%%%%%%
\end{document}
