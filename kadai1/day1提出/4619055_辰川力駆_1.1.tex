\documentclass[a4paper,12pt]{jarticle}
\newcommand{\vect}[1]{\mbox{\boldmath${#1}$}}
\setlength{\textwidth}{170mm}       % テキストの幅
\setlength{\textheight}{260mm}      % テキストの高さ
\setlength{\oddsidemargin}{-5mm}    % 偶数ページの左マージン
\setlength{\evensidemargin}{0mm}    % 奇数ページの左マージン
\setlength{\topmargin}{-25mm}       % 上のマージン
\usepackage{listings,jlisting}
\usepackage{multirow}
\usepackage[dvipdfmx]{graphicx}
\lstset{
    frame={tblr},
    basicstyle={\footnotesize},
    tabsize={4},
}
\begin{document}
\begin{center}
    学籍番号: \underline{4619055},氏名: \underline{辰川力駆}
\end{center}

\underline{\Large 演習課題 : コマンドの調査}

各コマンドのオプションを各自で意味を調査せよ。

\begin{itemize}
    \item mkdir・・・ディレクトリを新規作成するコマンド
          \begin{lstlisting}
    $ mkdir ~/textprocessing
        \end{lstlisting}
    \item cp・・・ファイルやディレクトリをコピーするコマンド
          \begin{lstlisting}
    $ cp jexpch1/alice.txt jexpch1/alice2.txt
    $ cp -r jexpch1 jexpchi1_copy
            \end{lstlisting}
          \begin{table}[h]
              \begin{center}
                  \begin{tabular}{|c|c|} \hline
                      オプション          & 意味                                                     \\ \hline
                      \multirow{2}{*}{-r} & コピー元にディレクトリを指定した場合、再帰的にコピーする \\
                                          & (サブディレクトリも含めて)                             \\ \hline
                  \end{tabular}
              \end{center}
          \end{table}
    \item mv・・・ファイルやディレクトリを(移動する・名前を変更する)コマンド
          \begin{lstlisting}
    $ mv jexpch1/alice2.txt jexpch1/alice_copy.txt
    $ mv jexpch1 textprocessing
            \end{lstlisting}
    \item cd・・・カレントディレクトリを変更するコマンド
          \begin{lstlisting}
    $ cd ~/textprocessing/jexpch1
    $ cd ..
    $ ls .
    jexpch1
            \end{lstlisting}
    \item echo・・・メッセージなどを表示するコマンド
          \begin{lstlisting}
    $ echo "abcdefg"
    abcdefg
    $ echo -n "abcdefg"
    abcdefg%
        \end{lstlisting}
          \begin{table}[h]
              \begin{center}
                  \begin{tabular}{|c|c|} \hline
                      オプション & 意味                         \\ \hline
                      -n         & 出力文字の最後の改行をしない \\ \hline
                  \end{tabular}
              \end{center}
          \end{table}
    \item date・・・現在の日時を表示したり、設定し直したりするためのコマンド
          \begin{lstlisting}
    $ date
    2020年 7月13日 月曜日 14時46分26秒 JST
            \end{lstlisting}
          \clearpage
    \item grep・・・ファイルの中で文字列が含まれている行を表示するコマンド
          \begin{lstlisting}
    $ grep raspberry /usr/share/dict/words
    raspberry
    raspberrylike
    $ grep -o raspberry /usr/share/dict/words
    raspberry
    raspberry
    $ grep -v a jexpch1/sample1.txt
    b Bob
    g Guy
    $ grep Guy jexpch1/sample1.txt jexpch1/sample2.txt
    jexpch1/sample1.txt:g Guy
    jexpch1/sample2.txt:g Guy
                \end{lstlisting}
          \begin{table}[h]
              \begin{center}
                  \begin{tabular}{|c|c|} \hline
                      オプション & 意味                             \\ \hline
                      -o         & 検索結果に一致した文字を表示する \\ \hline
                      -v         & 一致しないものを検索する         \\ \hline
                      -f         & ディレクトリ内も検索対象とする   \\ \hline
                  \end{tabular}
              \end{center}
          \end{table}
    \item $[$・・・記述内容が真(0)か偽(1)か評価する。引数の最後は必ず$]$でなければならない。
          \begin{lstlisting}
    $ cd jexpch1
    $ [ -e sample1.txt ]
    $ echo $?
    0
    $ [ -e samplex.txt ]
    $ echo $?
    1
    $ [ ! -e sample1.txt ]
    $ echo $?
    1
    $ [ 5 -lt 10 ]
    $ echo $?
    0
    $ [ 5 -gt 10 ]
    $ echo $?
    1
    $ [ "abc" = "abc" ]
    $ echo $?
    0
    $ [ "abc" != "abe" ]
    $ echo $?
    0
    $ [ ! "abc" = "abe" ]
    $ echo $?
    0
                    \end{lstlisting}

          \clearpage
          \begin{table}[t]
              \begin{center}
                  \begin{tabular}{|c|c|} \hline
                      オプション & 意味                   \\ \hline
                      -e         & 存在するなら真         \\ \hline
                      !          & 結果の真と偽が逆になる \\ \hline
                      -lt        & より小さいなら真       \\ \hline
                      -gt        & より大きいなら真       \\ \hline
                      =          & 等しければ真           \\ \hline
                      !=         & 等しくなければ真       \\ \hline
                  \end{tabular}
              \end{center}
          \end{table}
    \item test・・・ $[$ コマンドと同じ
    \item touch・・・空のファイルを作成したり、時刻を変更するコマンド
          \begin{lstlisting}
    $ ls -lat jexpch1
    total 288
    drwxr-xr-x   3 tea  staff     96  7 13 14:33 ..
    drwxr-xr-x@ 18 tea  staff    576  7 13 14:32 .
    -rw-r--r--@  1 tea  staff  73529  5  8  2017 partial_words.txt
    -rw-r--r--@  1 tea  staff     15  5  8  2017 ex2.4.txt
    -rw-r--r--@  1 tea  staff     24  5  8  2017 output.txt
    -rw-r--r--@  1 tea  staff      0  5  8  2017 sample11.txt
    -rw-r--r--@  1 tea  staff     15  4 24  2017 sort_sample2.txt
    -rw-r--r--@  1 tea  staff     28  4 24  2017 abcd.txt
    -rw-r--r--@  1 tea  staff     70  4 24  2017 sort_sample1.txt
    -rw-r--r--@  1 tea  staff  11658  4 24  2017 alice.txt
    -rw-r--r--@  1 tea  staff   2622  4 24  2017 iamcat.txt
    -rw-r--r--@  1 tea  staff     56  4 24  2017 sample1.txt
    -rw-r--r--@  1 tea  staff     54  4 24  2017 sample2.txt
    -rw-r--r--@  1 tea  staff     36  4 24  2017 sample3.txt
    -rw-r--r--@  1 tea  staff     90  4 24  2017 sample4.txt
    -rw-r--r--@  1 tea  staff    150  4 24  2017 sample5.txt
    -rw-r--r--@  1 tea  staff     18  4 24  2017 sample6.txt
    $ touch jexpch1/alice3.txt
    $ touch jexpch1/alice_copy.txt
    $ ls -lat jexpch1
    total 288
    -rw-r--r--@  1 tea  staff  11658  7 13 15:58 alice_copy.txt
    drwxr-xr-x@ 19 tea  staff    608  7 13 15:57 .
    -rw-r--r--   1 tea  staff      0  7 13 15:57 alice3.txt
    drwxr-xr-x   3 tea  staff     96  7 13 14:33 ..
    -rw-r--r--@  1 tea  staff  73529  5  8  2017 partial_words.txt
    -rw-r--r--@  1 tea  staff     15  5  8  2017 ex2.4.txt
    -rw-r--r--@  1 tea  staff     24  5  8  2017 output.txt
    -rw-r--r--@  1 tea  staff      0  5  8  2017 sample11.txt
    -rw-r--r--@  1 tea  staff     15  4 24  2017 sort_sample2.txt
    -rw-r--r--@  1 tea  staff     28  4 24  2017 abcd.txt
    -rw-r--r--@  1 tea  staff     70  4 24  2017 sort_sample1.txt
    -rw-r--r--@  1 tea  staff  11658  4 24  2017 alice.txt
    -rw-r--r--@  1 tea  staff   2622  4 24  2017 iamcat.txt
    -rw-r--r--@  1 tea  staff     56  4 24  2017 sample1.txt
    -rw-r--r--@  1 tea  staff     54  4 24  2017 sample2.txt
    -rw-r--r--@  1 tea  staff     36  4 24  2017 sample3.txt
    -rw-r--r--@  1 tea  staff     90  4 24  2017 sample4.txt
    -rw-r--r--@  1 tea  staff    150  4 24  2017 sample5.txt
    -rw-r--r--@  1 tea  staff     18  4 24  2017 sample6.txt
                    \end{lstlisting}
          \clearpage
    \item seq・・・指定した数字の列を出力するコマンド
          \begin{lstlisting}
    $ seq 10
    1
    2
    3
    4
    5
    6
    7
    8
    9
    10
    $ seq 2 5
    2
    3
    4
    5
                    \end{lstlisting}
    \item cat・・・ファイルを連結・表示するためのコマンド
          \begin{lstlisting}
    $ cat jexpchi1/sample1.txt
    a Alice
    b Bob
    c Cathy
    d Daniel
    e Emilia
    f Francis
    g Guy
    $ cat -b jexpchi1/sample1.txt  
        1	a Alice
        2	b Bob
        3	c Cathy
        4	d Daniel
        5	e Emilia
        6	f Francis
        7	g Guy
    $ cat -n jexpchi1/sample1.txt  
        1	a Alice
        2	b Bob
        3	c Cathy
        4	d Daniel
        5	e Emilia
        6	f Francis
        7	g Guy
        \end{lstlisting}
          \begin{table}[h]
              \begin{center}
                  \begin{tabular}{|c|c|} \hline
                      オプション & 意味                                         \\ \hline
                      -b         & 行番号を付け加える(ただし空白行には付けない) \\ \hline
                      -n         & 行番号を付け加える                           \\ \hline
                  \end{tabular}
              \end{center}
          \end{table}

          \clearpage
    \item head・・・最初の数行を表示するコマンド
          \begin{lstlisting}
    $ head -n 3 jexpchi1/sample1.txt
    a Alice
    b Bob
    c Cathy
    $ head -3 jexpchi1/sample1.txt
    a Alice
    b Bob
    c Cathy
        \end{lstlisting}
          \begin{table}[h]
              \begin{center}
                  \begin{tabular}{|c|c|} \hline
                      オプション & 意味                             \\ \hline
                      -n         & 先頭から指定した行数のみ表示する \\ \hline
                  \end{tabular}
              \end{center}
          \end{table}
    \item tail・・・最後の数行を表示するコマンド
          \begin{lstlisting}
    $ tail -n 3 jexpchi1/sample1.txt
    e Emilia
    f Francis
    g Guy
    $ tail -3 jexpchi1/sample1.txt
    e Emilia
    f Francis
    g Guy
            \end{lstlisting}
    \item wc・・・テキストファイルの文字数や行数を数えるためのコマンド
          \begin{lstlisting}
    $ wc abcd.txt
        7       7      28 abcd.txt
    $ wc -c abcd.txt
        28 abcd.txt
    $ wc -l abcd.txt
         7 abcd.txt
                \end{lstlisting}
          \begin{table}[h]
              \begin{center}
                  \begin{tabular}{|c|c|} \hline
                      オプション & 意味               \\ \hline
                      -c         & バイト数を表示する \\ \hline
                      -l         & 行の数を表示する   \\ \hline
                  \end{tabular}
              \end{center}
          \end{table}
          \clearpage
    \item nkf・・・文字コードを変換する
          \begin{lstlisting}
    $ cd jexpchi1
    $ nkf -e iamcat.txt > e_iamcat.txt
    $ nkf -g e_iamcat.txt
    EUC-JP
    $ nkf -j iamcat.txt > j_iamcat.txt
    $ nkf -g j_iamcat.txt
    ISO-2022-JP
    $ nkf -s iamcat.txt > s_iamcat.txt
    $ nkf -g s_iamcat.txt
    Shift_JIS
    $ nkf -w iamcat.txt > w_iamcat.txt
    $ nkf -g w_iamcat.txt
    UTF-8
    $ nkf -Lu iamcat.txt > lu_iamcat.txt
    $ nkf -g lu_iamcat.txt
    UTF-8
    $ nkf -Lm iamcat.txt > lm_iamcat.txt
    $ nkf -g lm_iamcat.txt
    UTF-8
        \end{lstlisting}
          \begin{table}[h]
              \begin{center}
                  \begin{tabular}{|c|c|} \hline
                      オプション & 意味                                   \\ \hline
                      -j         & JISコードを出力する                    \\ \hline
                      -s         & シフトJISコードを出力する              \\ \hline
                      -e         & EUCコードを出力する                    \\ \hline
                      -w         & UTF-8コードを出力する                  \\ \hline
                      -Lu        & 改行をLFにする(UNIX系)               \\ \hline
                      -Lm        & 改行をCRにする(OS Xより前のmac OS系) \\ \hline
                      -g         & 自動判別の結果を出力する               \\ \hline
                  \end{tabular}
              \end{center}
          \end{table}
          \clearpage
    \item diff・・・2つのテキストファイルを比較し、異なる箇所を出力するコマンド
          \begin{lstlisting}
    $ cd jexpchi1
    $ diff sample1.txt sample2.txt
    1c1
    < a Alice
    ---
    > a Alfred
    4,5c4
    < d Daniel
    < e Emilia
    ---
    > d David
    7a7
    > k Kate
    $ diff -u sample1.txt sample2.txt
    --- sample1.txt	2020-07-13 16:25:14.000000000 +0900
    +++ sample2.txt	2020-07-13 16:25:14.000000000 +0900
    @@ -1,7 +1,7 @@
    -a Alice
    +a Alfred
     b Bob
     c Cathy
    -d Daniel
    -e Emilia
    +d David
     f Francis
     g Guy
    +k Kate
            \end{lstlisting}

          \begin{table}[h]
              \begin{center}
                  \begin{tabular}{|c|c|} \hline
                      オプション & 意味                                                        \\ \hline
                      -u         & 違いのある箇所を1つにまとめて、-記号と+記号で変更箇所を示す \\ \hline
                  \end{tabular}
              \end{center}
          \end{table}
    \item join・・・2つのテキストファイルを比較し、共通する項目がある行を連結するコマンド
          \begin{lstlisting}
    $ cd jexpchi1
    $ join sample1.txt sample2.txt
    a Alice Alfred
    b Bob Bob
    c Cathy Cathy
    d Daniel David
    f Francis Francis
    g Guy Guy
    $ join -1 2 -2 2 sample1.txt sample2.txt
    Bob b b
    Cathy c c
    Francis f f
    Guy g g
    $ join -a 2 -1 2 -2 2 sample1.txt sample2.txt
    a Alfred
    Bob b b
    Cathy c c
    d David
    Francis f f
    Guy g g
    k Kate
           \end{lstlisting}
          \clearpage
          \begin{table}[h]
              \begin{center}
                  \begin{tabular}{|c|c|} \hline
                      オプション & 意味                                               \\ \hline
                      -1         & 1つめに指定したファイルで一致に使う項目を指定する \\ \hline
                      -2         & 2つめに指定したファイルで一致に使う項目を指定する \\ \hline
                      -a         & 指定したファイルの行はすべて表示する               \\ \hline
                  \end{tabular}
              \end{center}
          \end{table}
    \item paste・・・各ファイルの行を結合するコマンド
          \begin{lstlisting}
    $ paste sample1.txt sample2.txt
    a Alice	a Alfred
    b Bob	b Bob
    c Cathy	c Cathy
    d Daniel	d David
    e Emilia	f Francis
    f Francis	g Guy
    g Guy	k Kate
    $ paste -d ' ' sample1.txt sample2.txt
    a Alice a Alfred
    b Bob b Bob
    c Cathy c Cathy
    d Daniel d David
    e Emilia f Francis
    f Francis g Guy
    g Guy k Kate
    $ paste -d ',' sample1.txt sample2.txt
    a Alice,a Alfred
    b Bob,b Bob
    c Cathy,c Cathy
    d Daniel,d David
    e Emilia,f Francis
    f Francis,g Guy
    g Guy,k Kate
               \end{lstlisting}
          \begin{table}[!h]
              \begin{center}
                  \begin{tabular}{|c|c|} \hline
                      オプション & 意味                 \\ \hline
                      -d         & 区切り文字を指定する \\ \hline
                  \end{tabular}
              \end{center}
          \end{table}
          \clearpage
    \item uniq・・・重複している行を取り除くコマンド
          \begin{lstlisting}
    $ cat < abcd.txt
    aaa
    bbb
    bbb
    bbb
    ddd
    ccc
    ddd
    $ uniq < abcd.txt
    aaa
    bbb
    ddd
    ccc
    ddd
    $ uniq -c < abcd.txt
        1 aaa
        3 bbb
        1 ddd
        1 ccc
        1 ddd
                   \end{lstlisting}
          \begin{table}[h]
              \begin{center}
                  \begin{tabular}{|c|c|} \hline
                      オプション & 意味                         \\ \hline
                      -c         & 各行の前に出現回数を出力する \\ \hline
                  \end{tabular}
              \end{center}
          \end{table}
          \clearpage
    \item sort・・・テキストファイルの内容を並べ替えるコマンド
          \begin{table}[h]
              \begin{center}
                  \begin{tabular}{cc}
                      \begin{minipage}{0.45\hsize}
                          \begin{lstlisting}[frame={tblr}]
    $ cd jexpchi1
    $ cat sort_sample1.txt
    5 c Cathy
    6 e Emilia
    5 a Alice
    3 g Guy
    7 f Francis
    3 b Bob
    6 d Daniel
    $ sort sort_sample1.txt
    3 b Bob
    3 g Guy
    5 a Alice
    5 c Cathy
    6 d Daniel
    6 e Emilia
    7 f Francis
    $ sort -k 2 -r sort_sample1.txt
    3 g Guy
    7 f Francis
    6 e Emilia
    6 d Daniel
    5 c Cathy
    3 b Bob
    5 a Alice
    $ sort -r sort_sample1.txt
    7 f Francis
    6 e Emilia
    6 d Daniel
    5 c Cathy
    5 a Alice
    3 g Guy
    3 b Bob
                       \end{lstlisting}
                      \end{minipage}
                      \begin{minipage}{0.47\hsize}
                          \begin{lstlisting}[frame={tbr}]
    $ cat sort_sample2.txt
    128
    512
    32
    4
    8
    $ sort sort_sample2.txt
    128
    32
    4
    512
    8
    $ sort -n sort_sample2.txt
    4
    8
    32
    128
    512
            \end{lstlisting}
                      \end{minipage}
                  \end{tabular}
              \end{center}
          \end{table}
          \begin{table}[!h]
              \begin{center}
                  \begin{tabular}{|c|c|} \hline
                      オプション & 意味                     \\ \hline
                      -k         & キーを指定して並べ替える \\ \hline
                      -r         & 降順で並べ替える         \\ \hline
                      -n         & 数値として並べ替える     \\ \hline
                  \end{tabular}
              \end{center}
          \end{table}
    \item expr・・・式を評価する
          \begin{lstlisting}
    $ expr 3 + 4
    7
    $ expr \( 7 \* 13 \) / 5
    18
    $ expr 7 \* 13
    91
    $ expr 7*13
    zsh: no matches found: 7*13
                           \end{lstlisting}
          \clearpage
    \item find・・・ディレクト階層をたどって条件を満たすファイルを検索する
          \begin{lstlisting}
    $ find jexpchi1 -type f
    jexpchi1/sample11.txt
    jexpchi1/alice.txt
    jexpchi1/alice3.txt
    jexpchi1/ex2.4.txt
    jexpchi1/alice2.txt
    jexpchi1/partial_words.txt
    jexpchi1/abcd.txt
    jexpchi1/sample6.txt
    jexpchi1/sample4.txt
    jexpchi1/iamcat.txt
    jexpchi1/sample5.txt
    jexpchi1/sample1.txt
    jexpchi1/sort_sample1.txt
    jexpchi1/output.txt
    jexpchi1/sample2.txt
    jexpchi1/sort_sample2.txt
    jexpchi1/sample3.txt
    $ find . -type f
    ./jexpchi1/sample11.txt
    ./jexpchi1/alice.txt
    ./jexpchi1/alice3.txt
    ./jexpchi1/ex2.4.txt
    ./jexpchi1/alice2.txt
    ./jexpchi1/partial_words.txt
    ./jexpchi1/abcd.txt
    ./jexpchi1/sample6.txt
    ./jexpchi1/sample4.txt
    ./jexpchi1/iamcat.txt
    ./jexpchi1/sample5.txt
    ./jexpchi1/sample1.txt
    ./jexpchi1/sort_sample1.txt
    ./jexpchi1/output.txt
    ./jexpchi1/sample2.txt
    ./jexpchi1/sort_sample2.txt
    ./jexpchi1/sample3.txt
    ./.DS_Store
    ./sample_dir/dir2/dummy1.txt
    ./sample_dir/dir2/dummy2.txt
    ./sample_dir/dir4/dummy1.txt
    ./sample_dir/dir4/dummy2.txt
    ./sample_dir/dir3/sample1.txt
    ./sample_dir/dir3/sample2.txt
    ./sample_dir/dir1/dummy1.txt
    ./sample_dir/dir1/dummy2.txt
    $ find . -mtime 1

    $ find jexpchi1 -name 'sample1.txt'
    jexpchi1/sample1.txt
                           \end{lstlisting}
          \begin{table}[h]
              \begin{center}
                  \begin{tabular}{|c|c|}
                      \hline
                      オプション & 意味                                                   \\ \hline
                      -mtime     & 任意の日数に更新されたファイルやディレクトリを検索する \\      \hline
                      -type      & 対象とするものをディレクトリかファイルか指定できる     \\      \hline
                      -name      & ファイルやディレクトリ名の一部のみ指定した検索ができる \\      \hline
                  \end{tabular}
              \end{center}
          \end{table}
          \clearpage
    \item basename・・・ファイル名からディレクトリと接尾辞を取り除く
          \begin{lstlisting}
    $ basename jexpchi1/sample1.txt
    sample1.txt
    $ basename -s '.txt' jexpchi1/sample1.txt
    sample1
                               \end{lstlisting}
          \begin{table}[h]
              \begin{center}
                  \begin{tabular}{|c|c|}
                      \hline
                      オプション & 意味                                         \\ \hline
                      -s         & ファイル名の末尾から指定した文字列を削除する \\      \hline
                  \end{tabular}
              \end{center}
          \end{table}

    \item tr・・・文字の変換や削除を行うコマンド
          \begin{lstlisting}
    $ cd jexpchi1
    $ tr 'l' 'L' < sample1.txt
    a ALice
    b Bob
    c Cathy
    d DanieL
    e EmiLia
    f Francis
    g Guy
    $ tr -d ' ' < sample1.txt
    aAlice
    bBob
    cCathy
    dDaniel
    eEmilia
    fFrancis
    gGuy
    $ tr -d '\n' < sample1.txt
    a Aliceb Bobc Cathyd Daniele Emiliaf Francisg Guy%
                           \end{lstlisting}
          \begin{table}[h]
              \begin{center}
                  \begin{tabular}{|c|c|}
                      \hline
                      オプション & 意味                                       \\ \hline
                      -d         & 最初の引数の文字を削除する(置換は行わない) \\      \hline
                  \end{tabular}
              \end{center}
          \end{table}
    \item wget・・・HTTPアクセスをしてコンテンツにファイルを保存するコマンド
          \begin{lstlisting}
    $ wget http://www.ms.kagu.tus.ac.jp
    --2020-07-14 19:56:58--  http://www.ms.kagu.tus.ac.jp/
    www.ms.kagu.tus.ac.jp (www.ms.kagu.tus.ac.jp) をDNSに問いあわせています
    ... 133.31.57.12
    www.ms.kagu.tus.ac.jp (www.ms.kagu.tus.ac.jp)|133.31.57.12|:80 に接続
    しています... 接続しました。
    HTTP による接続要求を送信しました、応答を待っています... 200 OK
    長さ: 特定できません [text/html]
    `index.html' に保存中
    
        [ <=>                                       
                                                       
        ] 33,514      --.-K/s 時間 0s
    
    2020-07-14 19:56:59 (228 MB/s) - `index.html' へ保存終了 [33514]
    
                           \end{lstlisting}
          \clearpage
    \item locale・・・現在のロケールを確認する
          \begin{lstlisting}
    $ locate
    locate: 探すパターンが指定されていません
    $ locate -a 
    locate: 無効なオプション -- 'a'
                           \end{lstlisting}
\end{itemize}

\end{document}