
\renewcommand{\thetable}{\arabic{section}.\arabic{table}}
\renewcommand{\theequation}{\arabic{section}.\arabic{subsection}.\arabic{equation}}

\vspace{7mm}
\noindent
・考察 \vspace{2mm}

J-K フリップ・フロップは,R-Sフリップ・フロップとは異なり,禁止入力がないため,扱いやすい記憶素子であると考えられる.


また,J-Kフリップ・フロップ回路は,トリガパルスを用いることで,出力のタイミングを自由に決めることができるという利点がある.この性質を用いると,0 から $N-1$ までカウントし
,カウントが $N$ になった時点でカウントを0にリセットする $N$ 進アップカウンタをつくることができる.J-Kフリップ・フロップ回路ではトリガパルスの電圧下降時にトリガされることから,下位からの桁上げを下位ビットの $1 \rightarrow 0$ の反転(すなわち,下位ビットの電圧降下)により検知し,その電圧降下をトリガとして用いることによって自分のビットを反転させる.これにより,10進で1カウントアップすることができる.また,各ビットの出力を並べた値が $N_{(10)}$ に対応する2進数に達した瞬間にだけ出力が ``1'' となる回路を構成し,その出力をすべてのJ-Kフリップ・フロップの $PC$ に用いることで値を ``0'' にリセットすることができる.このようにして $N$ 進カウンタをつくることができる.
\newpage
\subsection{NAND回路による組み合わせ回路の構成}

組み合わせ回路は全てNAND回路で構成できることを示す.基本素子 AND,OR,NOT,NORがNANDのみで構成できることが分かれば十分であるから,これを例を挙げながら示していくことにする.なお,NANDの真理値表は表\ref{tbl4}の通りである.

\vspace{7mm}
\noindent
(1) NOT回路 \vspace{2mm}

図3.B(a)のように,NAND素子に同一の ``A'' 2つを入力することにより,次の真理値表\ref{tbl01}が得られ,$\overline{A} = \overline{A \cdot A}$であることが分かる.したがって,NOT回路はNAND素子1つで実現することができる.実際,$X = X + X$ が成り立つこととド・モルガンの定理を用いれば,
$$
    \overline{A \cdot A} = \overline{A} + \overline{A} = \overline{A}
$$
となる.

\begin{table}[!h]
    \caption{$\overline{A},\;\overline{A \cdot A}$ の真理値表}
    \label{tbl01}
    \begin{center}
        \begin{tabular}{|c|c||c|c|}
            \hline
            $A$ & $A$ & $\overline{A}$ & $\overline{A \cdot A}$ \\
            \hhline{|=|=#=|=|}
            0   & 0   & 1              & 1                      \\
            \hline
            1   & 1   & 0              & 0                      \\
            \hline
        \end{tabular}
    \end{center}
\end{table}

\vspace{7mm}
\noindent
(2) AND回路 \vspace{2mm}

$A \cdot B = \overline{\overline{A \cdot B}}$ であることを用いれば,NAND素子の出力 $Y_1 = \overline{A \cdot B}$ をNOT素子に入力することで,AND回路を構成することができる.したがって,(1)を考慮すると,図3.B(b)のようにNAND回路を2つ用いて論理回路を構成すればよい.

\begin{table}[!h]
    \caption{$A \cdot B,\; \overline{\overline{A \cdot B}}$ の真理値表}
    \label{tbl02}
    \begin{center}
        \begin{tabular}{|c|c||c|c|c|}
            \hline
            $A$ & $B$ & $A \cdot B$ & $Y_1 = \overline{A \cdot B}$ & $\overline{Y_1}$ \\
            \hhline{|=|=#=|=|=|}
            0   & 0   & 0           & 1                            & 0                \\
            \hline
            0   & 1   & 0           & 1                            & 0                \\
            \hline
            1   & 0   & 0           & 1                            & 0                \\
            \hline
            1   & 1   & 1           & 0                            & 1                \\
            \hline
        \end{tabular}
    \end{center}
\end{table}

\vspace{7mm}
\noindent
(3) OR回路 \vspace{2mm}

ド・モルガンの定理より,$A + B = \overline{\overline{A + B}} = \overline{\overline{A} \cdot \overline{B}}$ であることを用いれば,入力 $A$,$B$ をNOT素子に入力し,その出力をNAND素子に入力することでOR回路を構成することができる.具体的には,(1)を考慮して,図3.B(c)のようにNAND回路を3つ用いて論理回路を構成すればよい.

\newpage
\begin{table}[!h]
    \caption{$A + B,\; \overline{\overline{A} \cdot \overline{B}}$ の真理値表}
    \label{tbl03}
    \begin{center}
        \begin{tabular}{|c|c||c|c|c|c|c|}
            \hline
            $A$ & $B$ & $A + B$ & $\overline{A}$ & $\overline{B}$ & $\overline{A} \cdot \overline{B}$ & $\overline{\overline{A} \cdot \overline{B}}$ \\
            \hhline{|=|=#=|=|=|=|=|}
            0   & 0   & 0       & 1              & 1              & 1                                 & 0                                            \\
            \hline
            0   & 1   & 1       & 1              & 0              & 0                                 & 1                                            \\
            \hline
            1   & 0   & 1       & 0              & 1              & 0                                 & 1                                            \\
            \hline
            1   & 1   & 1       & 0              & 0              & 0                                 & 1                                            \\
            \hline
        \end{tabular}
    \end{center}
\end{table}

\vspace{7mm}
\noindent
(4) NOR回路 \vspace{2mm}

(3)で構成したOR回路の出力を(1)で構成したNOT素子に入力することで,NOR回路を構成することができる.具体的には,図3.B(d)のようにNAND素子を4つ用いて論理回路を構成すればよい.

\vspace{7mm}

以上により,基本素子がすべてNAND素子のみで構成可能であることが分かった.これにより,基本素子で構成されるすべての組み合わせ回路についても,NAND素子のみで構成可能であることが示された.

\subsection{2値論理における減算の理論}

減算を行う回路には,加算と同様に,半減算器と全減算器がある.
2値論理における減算には図3.C(a)に示すような4つの基本演算がある.$0 - 1$ のときはその桁から引けないので,1つ上の桁から1を借りるという操作が必要である.$D = A - B$,及び桁借り $B_O$ として,半減算器のシンボルと真理値表はそれぞれ,図3.C(a),表\ref{tbl04}のようになる.

\begin{table}[!h]
    \caption{半減算器の真理値表}
    \label{tbl04}
    \begin{center}
        \begin{tabular}{|c|c||c|c|}
            \hline
            $A$ & $B$ & $B_O$ & $D$ \\
            \hhline{|=|=#=|=|}
            0   & 0   & 0     & 0   \\
            \hline
            0   & 1   & 1     & 1   \\
            \hline
            1   & 0   & 0     & 1   \\
            \hline
            1   & 1   & 0     & 0   \\
            \hline
        \end{tabular}
    \end{center}
\end{table}

真理値表から図3.C(c)(d)のような半減算回路が構成できる.また,加算と同様,半減算回路を2つ用いることで,桁借りを考慮した全減算回路を作ることができる(図3.D).

また,$A - B = A + (-B)$ であることを利用して,補数表示した2つの数値を加算するという方法で減算を実現することも可能である.

\newpage
\subsection{同期・非同期カウンタの回路と動作原理}

非同期式4進カウンタと同期式4進カウンタの回路を図3.Eに示した.これらのカウンタは,$Q_1 Q_0 = \in \{ 00, 01, 10, 11 \}$ の値をクロックパルスが与えられるごとにカウントアップする回路である.非同期カウンタでは,$Q_0$ の値が $1 \rightarrow 0$ に下がったときに $Q_1$ の値が反転することを利用し,$Q_0$ を2桁目の加算のクロックに用いている.一方同期カウンタでは,$Q_0, Q_1$ ともに同一のクロックパルスを用い,また $J_1$,$K_1$ に $Q_0$ を入力することにより $Q_0$,$Q_1$ に対して演算を行っている.

\section{結論}

本実験では,論理回路の基本的な素子の動作とその応用を理解した.とりわけ,加減算回路などは計算機設計などでは欠かせない分野であり,さらに理解を深める必要があると思われる.

\begin{thebibliography}{9}
    \bibitem{bib1} 東京理科大学工学部情報工学科 (2017) 『情報工学実験1:平成29年度』
    \bibitem{bib2} 大類重範 (2010-2017) 『ディジタル電子回路』 日本理工出版会
    \bibitem{bib3} 松本洋平 『情報処理基礎論』 [online]\underline{http://www2.kaiyodai.ac.jp/~matumoto/lecture02/}
    2017年5月28日最終閲覧
    \bibitem{bib4} ``Camp Network'' $\;$   [online]\underline{http://capm-network.com/} $\;$ 2017年5月28日最終閲覧
    \bibitem{bib5} 『エンコーダとデコーダ』$\;$ [online]\underline{http://home.a00.itscom.net/hatada/dc2/chap15/decoder.html} $\;\;$ 2017年5月28日最終閲覧
    \bibitem{bib6} 『RSフリップフロップの禁止(不定)について』\\ $\;$  [online]\underline{http://shusaku721-bibou6.seesaa.net/article/441811548.html} $\;$ 2017年5月28日最終閲覧
\end{thebibliography}

\end{document}
