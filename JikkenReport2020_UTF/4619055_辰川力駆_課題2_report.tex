%4619055 課題2
\documentclass[12pt]{jarticle}
\usepackage{TUSIreport}
\usepackage{otf}
\usepackage{listings,jlisting}
\usepackage{color}
\usepackage{fancybox,ascmac}
\usepackage{url}
\lstset{
language=[x86masm]Assembler,
backgroundcolor={\color[gray]{.85}},
basicstyle={\small},
identifierstyle={\small},
commentstyle={\small\ttfamily \color[rgb]{0,0.5,0}},
keywordstyle={\small\bfseries \color[rgb]{1,0,0}},
ndkeywordstyle={\small},
stringstyle={\small\ttfamily \color[rgb]{0,0,1}},
frame={tb},
breaklines=true,
columns=[l]{fullflexible},
numbers=left,
xrightmargin=0zw,
xleftmargin=3zw,
numberstyle={\scriptsize},
stepnumber=1,
numbersep=1zw,
morecomment=[l]{//}
}
%%%%%%%%%%%%%%%%%%
\begin{document}
%%%%%%%%%%%%%%%%%%%%%%%%%%%%%%%%%%%%%%%%%%%%%%%%%%%%%%%%
% 表紙を出力する場合は,\提出者と\共同実験者をいれる
% \提出者{科目名}{課題名}{提出年}{提出月}{提出日}{学籍番号}{氏名}
% \共同実験者{一人目}{二人目}{..}{..}{..}{..}{..}{八人目}
%%%%%%%%%%%%%%%%%%%%%%%%%%%%%%%%%%%%%%%%%%%%%%%%%%%%%%%
\提出者{情報工学実験1}{課題2 アセンブリ言語}
{2020}{6}{1}{4619055}{辰川力駆}
\共同実験者{}{}{}{}{}{}{}{}
\表紙出力

\section{実験の目的}
機械語と1対1に対応していながら、人間が読み書きしやすいように作られたアセンブリ言語を学ぶ。
アセンブリ言語を用いてPCのCPUであるIntel x86を機械語で直接操作することによって、
CPUや周辺機器の動作についての理解を深める。

\section{実験の内容}
アセンブリ言語に慣れるために、シミュレータを使って(擬似的な)レジスタの値やフラグの値をチェックする。
その次に、8086に対応したMS-DOSという16ビットOSを対象として、実際に8086をアセンブリ言語で操作する。

\section{実験環境}
\begin{itemize}
	\item MacBook Pro(16-inch,2019)
	      \begin{itemize}
		      \item ProductName: Mac OS X
		      \item ProductVersion:	10.15.5
		      \item BuildVersion: 19F101
		      \item プロセッサ: 2.6 GHz 6コアIntel Core i7
		      \item メモリ: 16 GB 2667 MHz DDR4
	      \end{itemize}
	\item NASM version: 2.14.02
	\item DOSBox version: 0.74-3
\end{itemize}

\section{結果・考察}

\subsection{レポート課題1}
\begin{itembox}[l]{課題1}
	本課題では ``シミュレータ" と ``エミュレータ" を用いて実験を行った。
	``シミュレータ" と ``エミュレータ" とは何かについて調査し、レポートにまとめよ。
	その際に ``シミュレータ" と ``エミュレータ" の違いを具体例を引用しながら説明せよ。
\end{itembox}

エミュレータは、1つのコンピュータシステムが他のシステムと同じように動作することを可能にするハードウェアまたはソフトウェアのことである。
シミュレータは、コンピュータのシステムなどを模擬的に再現する機能を持ったハードウェアまたはソフトウェアのことである。
つまり、全く同じ動作をするか模擬的であるかの違いである。

具体例あげると、本実験ではシミュレータとして 16bit Assembler Simulator を使用し、エミュレータとして DOSBoxを使用した。
16bit Assembler Simulatorの場合、文字列を出力したり、キー入力をすることはできない。
これに対し、DOSBoxはこれらができる他に、他のOS(Windows や Linux など)同様、システムコール全般が扱える。

\subsection{レポート課題2}
\begin{itembox}[l]{課題2}
	課題1-2で作成したプログラムについて、どのようなやり方で1から10までの和を求めたのかを、
	適宜ソースコードを引用しながら説明せよ。また、プログラム作成時に工夫した点・気をつけた点についても述べよ。

	レポートには作成したソースコード(sum\_4619055.asm)を載せること。また、ソースコードの各行に行番号を付けること。
\end{itembox}
\begin{lstlisting}[caption=sum\_4619055.asm,label=sum]
	MOV AX,0
	MOV CX,10
	label:
		ADD AX,CX
		LOOP label
	HLT
\end{lstlisting}

このソースコード\ref{sum}は、繰り返しジャンプ命令であるLOOPを用いてAXレジスタに1から10までを逆順に加算したプログラムである。

具体的には、2行目でLOOPをする回数をCXレジスタに保存し、次の行(3行目)にLOOPしたいラベル(今回はlabel)を書き、
そのラベル内で繰り返しジャンプ命令であるLOOPを書くことで、CXレジスタ回ラベルが実行されている。
LOOP命令では1回実行されるたびCXレジスタは$-1$されていくので、4行目でAXレジスタにCXレジスタを加えているが、$10,9,8,\cdots,1$という順番で加えられている。

工夫した点は、上記で述べたようにLOOPを用いたことである。
1から10を足すプログラムを作るだけであればLOOP演算を使わず、4行目のように加算命令ADDを用いて1から10を10行に渡り書くこともできたが、
コードが簡潔にならないのでLOOPを使用した。

\subsection{レポート課題3}
\begin{itembox}[l]{課題3}
	課題1-3において、AXレジスタに保存した数値を2進数で表現したときに1になるビットの数を求めるプログラムを作成した。
	どのようなやり方で1の数を数えたのか、適宜ソースコードを引用しながら説明せよ。また、プログラム作成時に工夫した点・気をつけた点についても述べよ。

	レポートには作成したソースコード(nbits\_4619055.asm)を載せること。また、ソースコードの各行に行番号を付けること。
\end{itembox}
\begin{lstlisting}[caption=nbits\_4619055.asm,label=nbits]
	MOV AX,80 ;;学籍番号は4619055なので55+25=80です。
	loop:
		MOV BX,AX
		AND BX,1 ;2進数表記で1の位が1ならばBXを1に、0ならばBXを0にします
		ADD DX,BX
		SHR AX,1
		CMP AX,0
		JE loopend
		CALL loop
	loopend: 
		HLT
\end{lstlisting}

このソースコード\ref{nbits}は、論理演算命令であるANDを用いて1の数を求め、DXレジスタに保存するプログラムである。

具体的には、1行目のコメント文でもあるように、80を2進数表記で表したときに1の数を求める。
AXレジスタをコピーしたBXレジスタを4行目でANDを用いて、BXレジスタの(2進数表記で)1の位が1ならばBXレジスタを1に、0ならばBXレジスタを0にしている。
その後、6行目でシフト命令であるSHRを用いることによりAXレジスタを(2進数表記で)右にずらすことと、
9行目でこの一連の動作を繰り返すことで、4行目の動作を(2進数表記で) $10,100,1000,\cdots$ の位の1の数を数えることができる。
このままでは、9行目によってずっと繰り返されるので7行目で比較する必要がなくなったら(AXレジスタが0になったら)終了する。

工夫した点としては、9行目でラベルを繰り返し行ったことである。
しかし、これだけでは永遠に繰り返しをしてしまうので、抜け出す条件を書くことに気をつけた。

\subsection{レポート課題4}
\begin{itembox}[l]{課題4}
	課題2-2において、じゃんけんプログラムを作成した。
	どのようなやり方でユーザーの手とコンピュータの手を決定したのか、適宜ソースコードを引用しながら説明せよ。
	また、本課題では、結果を画面に表示する順番を指定した。指定された順番で結果を表示するために工夫した点・気をつけた点についても述べよ。
	さらに、作成したプログラムの問題点を一つ取り上げ、その解決策について考察せよ。

	レポートには作成したソースコード(rps2\_4619055.asm)を載せること。また、ソースコードの各行に行番号を付けること。
\end{itembox}
\begin{lstlisting}[caption=rps2\_4619055.asm,label=rps2]
	org	0x100
	mov bx,0 ;;アドレステーブル参照のオフセットを示す変数
	
	call firstmsg ;;最初のメッセージを表示
	call inputfunc ;;文字列入力
	call rsltuser ;userの結果を表示
	call keywait ;;キー入力が入るとCPUのじゃんけんがきまる
	call rsltcpu ;CPUの結果を表示
	
	;;プログラムの終了
	mov ax,0x4c00
	int 0x21
	
	keywait:
		add bx,2
		cmp bx,4
		jle skip
		mov bx,0
	skip:
		;;キーボードの状態のチェック
		mov ah,0x06
		mov dl,0xff
		int 0x21
		jz keywait ;;キー入力がなければkeywaitへ戻る
		RET
		
	firstmsg:
		;;最初のメッセージを表示
		mov ah,0x09
		mov dx,msg 
		int 0x21
		RET
	
	rsltuser:
		;userの結果を表示
		add bx,ax
		add bx,ax
		and bx,0x0f
		mov ah,0x09
		mov dx,usrmsg 
		int 0x21
		mov dx,[handsign+bx]
		int 0x21
		mov bx,0
		RET
	
	rsltcpu:
		;CPUの結果を表示
		mov ah,0x09
		mov dx,cpumsg 
		int 0x21
		mov dx,[handsign+bx]
		int 0x21
		RET
		
	inputfunc:
		;;文字列入力
		mov ah,0x09
		mov dx,input ;inputメッセージを表示
		int 0x21
		mov ah,0x01 ;;文字列入力
		int 0x21
		mov ah,0x09
		mov dx,crlf ;;改行を表示
		int 0x21
		RET
	
	msg db 'Please Input Handsign (Rock->0, Paper->1, Scissors->2)',0x0d,0x0a,'$'
	input db 'INPUT:',0x0d,0x0a,'$'
	usrmsg db 'User: ','$'
	cpumsg db 'CPU: ','$'
	
	handsign dw rock,paper,scissors ;;アドレステーブル(ポインタが入った配列)
	rock db 'Rock',0x0d,0x0a,'$'
	paper db 'Paper',0x0d,0x0a,'$'
	scissors db 'Scissors',0x0d,0x0a,'$'
	crlf db 0x0d,0x0a,'$'
\end{lstlisting}

このソースコード\ref{rps2}は、

\subsection{レポート課題5}
\begin{itembox}[l]{課題5}
	fib.asm は、フィボナッチ数列を求めるプログラムである。
	フィボナッチ数列は、$F_0 = 0$、$F_1 = 1$ として、
	漸化式
	$$F_{n+2} = F_n + F_{n+1}  (n \geq 0)$$
	によって得られる数列である。
	fib.asmの場合は $F_0 = 2$,$F_1 = 3$ としている。
	このfib.asmには同じ処理が繰り返し記述されている箇所や非効率的に書かれた箇所が存在する。
	そこで、サブルーチン (CALL 命令・RET 命令) および LOOP 命令を用いて、
	以下のプログラムをより少ないコード量になるように書き換えよ。
	ただし、結果は DX レジスタに保存することとする。
	また、プログラムを実行した結果が変わらないようにすること。
	ファイル名は、fib\_4619055.asm とする。
	レポートには作成したソースコードを載せること。
	また、ソースコードの各行に行番号を付けること。

	レポートでは、作成したプログラムについて、どのような工夫をすることでソースコード
	をより短いコードに書き換えたのかについて述べよ。
	また、なぜ fib.asm では、9 つめのフィボナッチ数までしか求めていないのか、考察せよ。
	説明するときには適宜ソースコードを引用すること。
\end{itembox}

\begin{lstlisting}[caption=fib\_4619055.asm,label=fib]
	;; フィボナッチ数列を求める
	org 0x100
	 
	;; 1つめのフィボナッチ数を計算 (5)
	mov ah, 2
	mov al, 3
	add ah, al
	loop:   ;2つめ~9つめのフィボナッチ数を計算
		cmp al,128 ;;2つ前の数が128を超えたとき終了する
		jae end
		mov bl, ah
		mov ah, al
		mov al, bl
		add ah, al
		call loop
	 
	end:
		mov dh, 0
		mov dl, ah          ; 結果をDLレジスタに保存
		;; プログラム終了
		mov ah, 0x4c
		int 0x21
\end{lstlisting}


\section{結論}
アセンブリ言語から,CPU や周辺機器の動作が分かって,コンピュータが動作する仕組みを
本質的に理解した.ポーリング方式と割り込み方式の利点・欠点とそれぞれの用途が分かった.
セグメントレジスタに関しても,それぞれの役割が分かった.


%参考文献
\begin{thebibliography}{99}
	\label{sannkoubunnkenn_chapter}
	\bibitem[1]{rikadai}東京理科大学工学部情報工学科 情報工学実験1 2020年度
	東京理科大学工学部情報工学科出版
	\bibitem[2]{simulator}エミュレータとシミュレータの違いは何ですか - との差

	[URL]: \url{https://ja.strephonsays.com/what-is-the-difference-between-emulator-and-simulator}

	最終閲覧日:2020/6/2

	\bibitem[3]{}LaTeXで色付きソースコードを貼り付け

	[URL]: \url{http://yu00.hatenablog.com/entry/2015/05/14/214121}

	最終閲覧日:2020/6/2


\end{thebibliography}

\clearpage
\appendix
%%%%%%%%%%%%%%%%%%%%%%%%%%%%%%%%%%%%%%%%%%%%%%%%%%%%%%%
\end{document}
