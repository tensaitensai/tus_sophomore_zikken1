%4619055 課題5
\documentclass[12pt]{jarticle}
\usepackage{TUSIreport}
\usepackage{otf}
\usepackage{listings,jlisting}
\usepackage{color}
\usepackage{fancybox,ascmac}
\usepackage{amssymb, amsmath}
\usepackage{hhline}
\usepackage{multirow}
\usepackage{url}
\lstdefinestyle{lstC}{
    language={C},
    backgroundcolor={\color[gray]{.85}},
    basicstyle={\small},
    identifierstyle={\small},
    commentstyle={\small\ttfamily \color[rgb]{0,0.5,0}},
    keywordstyle={\small\bfseries \color[rgb]{1,0,0}},
    ndkeywordstyle={\small},
    stringstyle={\small\ttfamily \color[rgb]{0,0,1}},
    frame={tb},
    breaklines=true,
    columns=[l]{fullflexible},
    numbers=left,
    xrightmargin=0zw,
    xleftmargin=3zw,
    numberstyle={\scriptsize},
    stepnumber=1,
    numbersep=1zw,
    morecomment=[l]{//}
}
\lstdefinestyle{lstbash}{
	language={bash},
	showstringspaces={false},
	tabsize={4},
    breaklines=true,
    frame={tb},
    basicstyle={\footnotesize},
    backgroundcolor={\color[cmyk]{0.07,0.077,0,0.561}},
	commentstyle={\itshape \color[rgb]{0.0,0.6,0.0}},
	keywordstyle={\bfseries \color[cmyk]{1.0,0.0,0.0,0.3}},
    stringstyle={\ttfamily \color[cmyk]{0.0, 1.0, 0.0, 0.0}},
}
%%%%%%%%%%%%%%%%%%
\begin{document}
%%%%%%%%%%%%%%%%%%%%%%%%%%%%%%%%%%%%%%%%%%%%%%%%%%%%%%%%
% 表紙を出力する場合は,\提出者と\共同実験者をいれる
% \提出者{科目名}{課題名}{提出年}{提出月}{提出日}{学籍番号}{氏名}
% \共同実験者{一人目}{二人目}{..}{..}{..}{..}{..}{八人目}
%%%%%%%%%%%%%%%%%%%%%%%%%%%%%%%%%%%%%%%%%%%%%%%%%%%%%%%
\提出者{情報工学実験1}{課題5 システム・プログラミング}
{2020}{7}{6}{4619055}{辰川力駆}
\共同実験者{}{}{}{}{}{}{}{}
\追加実験者{}{}
\表紙出力

\section{実験目的}
本実験では、システムコールを用いたシステムプログラミングを学ぶと共に、
プロセスの基本概念、並行プログラミングの基礎を理解する。
ここで対象とするオペレーティングシステム(Operating System、OS)は、
POSIX(IEEE Std 1003.1)準拠の各種UNIXやLinuxなどUNIX系OSである。

\section{実験の原理(理論)}
\subsection{カーネル}
すべてのコンピュータには、OSと呼ばれる基礎的なプログラムの集合があり、
その集合の中で最も重要なプログラムがカーネル(Kernel)である。

カーネルの重要な役割は、
ハードウェア(デバイス)の制御およびユーザプログラムに対する実行環境の提供である。
MS-DOSなどのOSではユーザプログラムが直接デバイスを操作することが許されているが、
UNIX系のOSではユーザプログラムがデバイスを直接操作したり、
任意のメモリ位置へのアクセスすることが禁止されている。
同じ種類のデバイスでもメーカーが違えば細かい制御方法が異なるため、
カーネルはそのようなデバイス毎の詳細をユーザから隠蔽し、
代わりにデバイスを操作するための共通インターフェース(システムコール)を提供する。
ユーザプログラムとデバイスの間にカーネルが入ることの利点としては、
ユーザは各デバイスの制御方法の詳細を知る必要がないためプログラミングが容易となることや、
カーネルが処理を実行する前に処理要求の妥当性の確認ができるため、
システムのセキュリティを大幅に向上させられることが挙げられる。
また、カーネルが同じインターフェースを提供する限り同じプログラムが実行できるため、
プログラムの可搬性も向上させることができる。

\subsection{システムコール}
ハードウェアと直接やり取りができるのはカーネルだけに限定されているため、
ユーザプログラムでハードウェアを操作したいときは、
カーネルを通して間接的に操作することになる。
そのために使うのがシステムコールである。
プログラムからのシステムコールの呼び出し方の見た目は通常の関数と変わらないが、
カーネルに仕事させるためにはシステムコールを使わなければならない。
通常の関数とシステムコールの違いは、
それがカーネルに対する明示的な要求かどうかである。

\subsection{ライブラリ関数}
プログラムを作る際、システムコール以外にも様々なライブラリ関数が使用できる。
最も代表的なものとしては標準Cライブラリがある。
標準Cライブラリにはprintf()、exit()、strlen()などが含まれる。
ライブラリ関数もシステムコールを使って実装されている場合があり、
例えばprintf()はwrite()というシステムコールが使われている。
通常、各システムコールに対応するライブラリ関数が用意されているが、
逆に各ライブラリ関数に対してシステムコールが存在するわけではない。
算術関数などはシステムコールを使う必要がない。

関数の定義を調べるためにはmanコマンドが利用可能である。
manページはいくつかのセクションに分かれている。
例えばprintfにはコマンドとライブラリ関数があるため、それぞれ
\begin{lstlisting}[style = lstbash]
    $ man 1 printf   # ユーザコマンド
    $ man 3 printf   # ライブラリ関数
\end{lstlisting}
のようにセクションを指定して検索する必要がある。
以後、本実験では関数名の後ろにそれがシステムコールならば(2)、
ライブラリ関数ならば(3)というようにセクション番号をつけてどの関数を使用しているかわかるようにする。

\subsection{プロセス}
プロセスの考え方は、どのようなOSでも基本となる考え方であり、
プログラムの実体として定義されている。
プロセスは、プログラムを実行した際などに生成され、
場合によっては子プロセスをさらに生成し、処理が終わると最後は消滅する。
カーネルはプロセスにシステムの資源を割り当て、
状態を管理する必要があるので、
詳細は触れないがプロセスディスクリプタを用いてプロセスを管理している。
各プロセスには重複しない番号(プロセスID)が振られており、
ユーザはプロセスIDを使うことで、プロセスを一意に指定することができる。
新しく生成されたプロセスのプロセスIDは直前に割り振られたプロセスID$+1$となる。
プロセスIDが$0$のプロセスはシステム起動時に生成される特別なプロセスである。

\subsection{ファイル}
ファイルとはバイト列として構造化された情報の入れ物である。
UNIX系OSではファイルはすべて一つの木構造で階層的に管理されており、
カーネルはファイルの中身を関知しない。
枝分かれの部分はディレクトリと呼ばれる特別なファイルがあり、
その枝にぶら下がっている下位の階層のファイルを管理している。
ディレクトリの中で最も上位に位置するのをルートディレクトリと呼び、
ルートディレクトリから順番にディレクトリを辿ることで任意のファイルを指定することができる。
この木構造の中には、ディレクトリだけでなくテキストファイルやバイナリファイル、
シンボリックリンク、デバイスファイルが含まれる。これらは全てファイルとして扱われる。

一般ファイルは、テキストファイルやプログラムなどである。
カーネルはファイルの中身を単にバイト列として構造化されたものとして扱う。
このようなファイルのことをストリームファイルと呼ぶ。
デバイスファイルは、各種デバイスである。
デバイスファイルもストリームファイルとして一般ファイルと同じように扱うことができる。
プログラムからストリームファイルにアクセスするとき、
ファイルディスクリプタというものを使う。
ファイルディスクリプタはプログラムから見ると単なる整数であるが、
カーネル側ではオープンされたストリームファイルと対応づけられている。
通常はopen(2)を用いてファイルをオープンし、
対応するファイルディスクリプタを取得する必要があるが、
標準入力、標準出力、標準エラー出力の3つはプログラム実行時に自動的にオープンされる。
それぞれのファイルディスクリプタは0、1、2である。
通常、標準入力にはキーボードが、
標準出力および標準エラー出力にはディスプレイが割り当てられている。

\clearpage
\section{実験環境}
\begin{itemize}
    \item MacBook Pro(16-inch,2019)
          \begin{itemize}
              \item ProductName: Mac OS X
              \item ProductVersion:	10.15.5
              \item BuildVersion: 19F101
              \item プロセッサ: 2.6 GHz 6コアIntel Core i7
              \item メモリ: 16 GB 2667 MHz DDR4
          \end{itemize}
    \item gcc version : 9.3.0
\end{itemize}

\section{実験結果・考察}
\subsection{レポート課題1}
\begin{shadebox}
    fgetsの他にgetsという関数があるが、この関数は``絶対に"使ってはならない。
    getsの定義を調べ、使用してはならない理由について説明せよ。
\end{shadebox}
gets関数は char *gets(char *s)である。
標準入力から1行を読み込み、
改行コード\verb|\|nまたはEOFにたどり着くと、
ヌル文字\verb|\|0を追加してchar *型で返す。
入力した文字列が長くて用意している限界の配列の
文字数を超えた場合、割り当てられたメモリ範囲を超えて書き込みが行われ、
バッファオーバーフローとなる。
よってgets関数が引き起こす暴走が
非常に危険であるから``絶対に"使ってはならない。

これに対処する解決法としては、
getsを使わずに、代わりに fgets を使えばよい。

\subsection{レポート課題2}
\begin{shadebox}
    \fbox{問題\textbf{2-1}}$\sim$\fbox{問題\textbf{2-3}}の結果をそれぞれまとめて考察せよ。
\end{shadebox}

\begin{itembox}[l]{\fbox{問題\textbf{2-1}}}
    ソースコードread\_2\_1byte.cとソースコードread\_3\_fgetc.cのプログラムを用いて、
    様々なファイルの処理時間を計測して比較せよ。
    時間の計測にはtime関数を用い、読み込んだファイルは/dev/nullへリダイレクトせよ。
\end{itembox}
\clearpage
\begin{lstlisting}[caption=read\_2\_1byte.c,label=read,style = lstC]
    #include <stdio.h>
    #include <stdlib.h>
    #include <unistd.h>
    #include <sys/types.h>
    #include <sys/stat.h>
    #include <fcntl.h>
    #define NBUF 1 //buf size
    
    void die(const char *s)
    {
        perror(s);
        exit(1);
    }
    
    void cat(const char *path)
    {
        ssize_t n;
        unsigned char buf[NBUF];
        int fd = open(path, O_RDONLY);
        if (fd < 0)
            die(path);
        for (;;)
        {
            n = read(fd, buf, NBUF);
            if (n < 0)
                die(path);
            if (n == 0)
                break;
            if (write(STDOUT_FILENO, buf, n) < 0)
                die(path);
        }
        if (close(fd) < 0)
            die(path);
    }
    
    int main(int argc, char *argv[])
    {
        if (argc < 2)
        {
            fprintf(stderr, "usage:%s file\n", argv[0]);
            exit(1);
        }
        int i;
        for (i = 1; i < argc; ++i)
        {
            cat(argv[i]);
        }
        return 0;
    }
\end{lstlisting}
\clearpage
\begin{lstlisting}[caption=read\_3\_fgetc.c,label=read2, style=lstC]
    #include <stdio.h>
    #include <stdlib.h>
    
    void die(const char *s)
    {
        perror(s);
        exit(1);
    }
    
    void cat(const char *path)
    {
        FILE *f = fopen(path, "r");
        if (!f)
            die(path);
    
        int c;
        while ((c = fgetc(f)) != EOF)
        {
            if (putchar(c) < 0)
                die(path);
        }
    
        if (fclose(f))
            die(path);
    }
    
    int main(int argc, char *argv[])
    {
        if (argc < 2)
        {
            fprintf(stderr, "usage:%s file\n", argv[0]);
            exit(1);
        }
    
        int i;
        for (i = 1; i < argc; ++i)
        {
            cat(argv[i]);
        }
    
        return 0;
    }
\end{lstlisting}
\clearpage
\begin{table}[t]
    \caption{read\_2\_1byte.cとread\_3\_fgetc.cの処理時間}
    \begin{center}
        \begin{tabular}{|c||c|c|c|c|c|c|}
            \hline
            \multirow{2}{*}{ファイル名} & \multicolumn{3}{|c|}{read\_2\_1byte.c} & \multicolumn{3}{|c|}{read\_3\_fgetc.c}                                             \\
            \cline{2-7}
                                        & real                                   & user                                   & sys      & real     & user     & sys      \\
            \hhline{|=#===|===|}
            sample\_10.txt              & 0m0.006s                               & 0m0.002s                               & 0m0.002s & 0m0.005s & 0m0.001s & 0m0.002s \\
            \hline
            sample\_100.txt             & 0m0.007s                               & 0m0.002s                               & 0m0.003s & 0m0.005s & 0m0.001s & 0m0.002s \\
            \hline
            sample\_4096.txt            & 0m0.013s                               & 0m0.005s                               & 0m0.007s & 0m0.005s & 0m0.001s & 0m0.002s \\
            \hline
            sample\_8192.txt            & 0m0.024s                               & 0m0.007s                               & 0m0.013s & 0m0.006s & 0m0.002s & 0m0.002s \\
            \hline
            sample\_1000000.txt         & 0m1.325s                               & 0m0.498s                               & 0m0.791s & 0m0.038s & 0m0.027s & 0m0.003s \\
            \hline
            sample\_10000000.txt        & 0m12.755s                              & 0m4.850s                               & 0m7.616s & 0m0.189s & 0m0.174s & 0m0.007s \\
            \hline
        \end{tabular}
    \end{center}
    \label{tbl1}
\end{table}
問題2-1の結果は上記の表に示した。
この表により、fgetc(3)、putchar(3)を用いて実行したときの方が実行時間が短かったと分かる。
また、sample\_10.txtやsample\_100.txtなどは実行時間が早いので差が少ない。

\begin{itembox}[l]{\fbox{問題\textbf{2-2}}}
    ソースコードread\_2\_1byte.cのNBUFを2048に変更し、
    実行時間がどのように変化するか調べよ。
\end{itembox}
\begin{table}[h]
    \caption{read\_2\_1byte.cとread\_2\_2048byte.cの処理時間}
    \begin{center}
        \begin{tabular}{|c||c|c|c|c|c|c|}
            \hline
            \multirow{2}{*}{ファイル名} & \multicolumn{3}{|c|}{read\_2\_1byte.c} & \multicolumn{3}{|c|}{read\_2\_2048byte.c}                                             \\
            \cline{2-7}
                                        & real                                   & user                                      & sys      & real     & user     & sys      \\
            \hhline{|=#===|===|}
            sample\_10.txt              & 0m0.006s                               & 0m0.002s                                  & 0m0.002s & 0m0.006s & 0m0.002s & 0m0.002s \\
            \hline
            sample\_100.txt             & 0m0.007s                               & 0m0.002s                                  & 0m0.003s & 0m0.009s & 0m0.002s & 0m0.003s \\
            \hline
            sample\_4096.txt            & 0m0.013s                               & 0m0.005s                                  & 0m0.007s & 0m0.009s & 0m0.002s & 0m0.003s \\
            \hline
            sample\_8192.txt            & 0m0.024s                               & 0m0.007s                                  & 0m0.013s & 0m0.007s & 0m0.002s & 0m0.003s \\
            \hline
            sample\_1000000.txt         & 0m1.325s                               & 0m0.498s                                  & 0m0.791s & 0m0.012s & 0m0.002s & 0m0.004s \\
            \hline
            sample\_10000000.txt        & 0m12.755s                              & 0m4.850s                                  & 0m7.616s & 0m0.028s & 0m0.005s & 0m0.013s \\
            \hline
        \end{tabular}
    \end{center}
    \label{tbl2}
\end{table}

問題2-2の結果は上記の表に示した。
この表により、NBUFを2048にして実行したときの方が実行時間が短かったと分かる。
また、問題2-1より明らかに早くなっているので、ライブラリ関数を用いるより、
システムコールを用いる方が早くなることが分かった。

\clearpage
\begin{itembox}[l]{\fbox{問題\textbf{2-3}}}
    ソースコードread\_2\_1byte.c、ソースコードread\_3\_fgetc.c、
    問題2-2で修正したコード、ソースコードread\_3\_fgets.cにおいて、
    システムコールがどのように呼ばれているか確認せよ。
\end{itembox}
\begin{lstlisting}[caption=read\_3\_fgets.c,label=read4, style=lstC]
    #include <stdio.h>
    #include <stdlib.h>
    #include <unistd.h>
    #include <sys/types.h>
    #include <sys/stat.h>
    #include <fcntl.h>
    
    #define NBUF 2048 //buf size
    
    void die(const char *s)
    {
        perror(s);
        exit(1);
    }
    
    void cat(const char *path)
    {
        unsigned char buf[NBUF];
        FILE *f = fopen(path, "r");
        if (!f)
            die(path);
        int c;
        while (fgets(buf, NBUF, f) != NULL)
        {
            printf("%s\n", buf);
        }
        if (fclose(f))
            die(path);
    }
    
    int main(int argc, char *argv[])
    {
        if (argc < 2)
        {
            fprintf(stderr, "usage:%s file\n", argv[0]);
            exit(1);
        }
        int i;
        for (i = 1; i < argc; ++i)
        {
            cat(argv[i]);
        }
    
        return 0;
    }
\end{lstlisting}
\clearpage
\fbox{ソースコードread\_2\_1byte.cについて}

以下にstraceの結果を示す。これを見ると分かるように、NBUFを1にしているので、
一文字ずつread、writeを繰り返している。
\begin{lstlisting}[style = lstbash]
    $ strace -e trace=open,read,write,close ./read_2_1byte sample_10.txt > /dev/null
    open("/etc/ld.so.cache", O_RDONLY|O_CLOEXEC) = 3
    close(3)                                = 0
    open("/lib64/libc.so.6", O_RDONLY|O_CLOEXEC) = 3
    read(3, "\177ELF\2\1\1\3\0\0\0\0\0\0\0\0\3\0>\0\1\0\0\0\340$\2\0\0\0\0\0"..., 832) = 832
    close(3)                                = 0
    open("sample_10.txt", O_RDONLY)         = 3
    read(3, "a", 1)                         = 1
    write(1, "a", 1)                        = 1
    read(3, "a", 1)                         = 1
    write(1, "a", 1)                        = 1
    read(3, "a", 1)                         = 1
    write(1, "a", 1)                        = 1
    read(3, "a", 1)                         = 1
    write(1, "a", 1)                        = 1
    read(3, "a", 1)                         = 1
    write(1, "a", 1)                        = 1
    read(3, "a", 1)                         = 1
    write(1, "a", 1)                        = 1
    read(3, "a", 1)                         = 1
    write(1, "a", 1)                        = 1
    read(3, "a", 1)                         = 1
    write(1, "a", 1)                        = 1
    read(3, "a", 1)                         = 1
    write(1, "a", 1)                        = 1
    read(3, "\n", 1)                        = 1
    write(1, "\n", 1)                       = 1
    read(3, "", 1)                          = 0
    close(3)                                = 0
    +++ exited with 0 +++
\end{lstlisting}

\fbox{問題2-2で修正したコードについて}

以下にstraceの結果を示す。
これは、ソースコードread\_2\_1byte.cの時と比べると、
1回のreadとwhiteで読み取っていることが分かる。
これは前述で述べたが、
NBUFが2048に変更したからである。
\begin{lstlisting}[style = lstbash]
    $ strace -e trace=open,read,write,close ./read_2_2048byte sample_10.txt > /dev/null
    open("/etc/ld.so.cache", O_RDONLY|O_CLOEXEC) = 3
    close(3)                                = 0
    open("/lib64/libc.so.6", O_RDONLY|O_CLOEXEC) = 3
    read(3, "\177ELF\2\1\1\3\0\0\0\0\0\0\0\0\3\0>\0\1\0\0\0\340$\2\0\0\0\0\0"..., 832) = 832
    close(3)                                = 0
    open("sample_10.txt", O_RDONLY)         = 3
    read(3, "aaaaaaaaa\n", 2048)            = 10
    write(1, "aaaaaaaaa\n", 10)             = 10
    read(3, "", 2048)                       = 0
    close(3)                                = 0
    +++ exited with 0 +++
\end{lstlisting}

\clearpage
\fbox{ソースコードread\_3\_fgetc.cについて}

以下にstraceの結果を示す。
NBUFを2048にしたのと比べると、
最後に行うreadとwriteとcloseの順番が違うのが分かる。
ソースコードread\_3\_fgetc.cでは、最後にwriteされている。
\begin{lstlisting}[style = lstbash]
    $ strace -e trace=open,read,write,close ./read_3_fgetc sample_10.txt > /dev/null
    open("/etc/ld.so.cache", O_RDONLY|O_CLOEXEC) = 3
    close(3)                                = 0
    open("/lib64/libc.so.6", O_RDONLY|O_CLOEXEC) = 3
    read(3, "\177ELF\2\1\1\3\0\0\0\0\0\0\0\0\3\0>\0\1\0\0\0\340$\2\0\0\0\0\0"..., 832) = 832
    close(3)                                = 0
    open("sample_10.txt", O_RDONLY)         = 3
    read(3, "aaaaaaaaa\n", 8192)            = 10
    read(3, "", 8192)                       = 0
    close(3)                                = 0
    write(1, "aaaaaaaaa\n", 10)             = 10
    +++ exited with 0 +++
\end{lstlisting}

\fbox{ソースコードread\_3\_fgets.cについて}

以下にstraceの結果を示す。
ソースコードread\_3\_fgetc.cと比べると、
最後のwriteの$\backslash n$が一つ多い。
これはソースコードread\_3\_fgets.c内のprintfで
自分で$\backslash n$を行っているからだと考える。
\begin{lstlisting}[style = lstbash]
    $ strace -e trace=open,read,write,close ./read_3_fgets sample_10.txt > /dev/null
    open("/etc/ld.so.cache", O_RDONLY|O_CLOEXEC) = 3
    close(3)                                = 0
    open("/lib64/libc.so.6", O_RDONLY|O_CLOEXEC) = 3
    read(3, "\177ELF\2\1\1\3\0\0\0\0\0\0\0\0\3\0>\0\1\0\0\0\340$\2\0\0\0\0\0"..., 832) = 832
    close(3)                                = 0
    open("sample_10.txt", O_RDONLY)         = 3
    read(3, "aaaaaaaaa\n", 8192)            = 10
    read(3, "", 8192)                       = 0
    close(3)                                = 0
    write(1, "aaaaaaaaa\n\n", 11)           = 11
    +++ exited with 0 +++
\end{lstlisting}

\clearpage
\subsection{レポート課題3}
\begin{shadebox}
    \fbox{演習\textbf{2}}で作ったプログラム(read\_3\_fgets.c)を基に、標準入力から読み込み、
    その行数を出力するコマンドを作成せよ(「wc -l」と同等)。
\end{shadebox}
\begin{lstlisting}[caption=kadai3.c,label=kadai3, style=lstC]
    #include <stdio.h>
    #include <stdlib.h>
    #include <unistd.h>
    #include <sys/types.h>
    #include <sys/stat.h>
    #include <fcntl.h>
    
    #define NBUF 2048 //buf size
    
    void die(const char *s)
    {
        perror(s);
        exit(1);
    }
    
    int totalcnt = 0; //totalの行数を数える変数
    void wc_l(const char *path)
    {
        int cnt = 0; // 行数を数える変数
        unsigned char buf[NBUF];
    
        FILE *f = fopen(path, "r");
        if (!f)
            die(path);
        int c;
        while (fgets(buf, NBUF, f) != NULL)
        {
            cnt++; // 一行読み込むごとにカウントする
        }
        if (fclose(f))
            die(path);
    
        printf("%8d %s\n", cnt, path); // ファイル名と行数を出力する
        totalcnt += cnt;               // 行数をtotalに足す
    }
    
    int main(int argc, char *argv[])
    {
        if (argc < 2)
        {
            fprintf(stderr, "usage:%s file\n", argv[0]);
            exit(1);
        }
        int i;
        if (argc == 2) // 一つのtxtしか読み込まない場合は行数のtotalがいらない
        {
            for (i = 1; i < argc; ++i)
            {
                wc_l(argv[i]);
            }
        }
        else // 複数の行数を読み込んだ場合は最後にtotalの行数を出力する。
        {
            for (i = 1; i < argc; ++i)
            {
                wc_l(argv[i]);
            }
            printf("%8d %s\n", totalcnt, "total"); //totalの行数を表示
        }
    
        return 0;
    }
\end{lstlisting}

上記に作成したソースコード(kadai3.c)を示した。
まず、wc -lを実行してみると、
1つだけをカウントするときは1つだけの結果を表示するが、
複数ある場合はそれぞれの結果を表示した上でさらに合計の行数であるtotalを表示していた。
また、結果の表示の仕方は数字8桁分を確保して半角空白をあけた後、
カウントしたファイルを表示という形であったので、
表示の仕方はprintf(``\%8d \%s$\backslash n$");でいいとすぐ推測できた。

これを踏まえて、まずはtotal無しを作成した。
変数cntを用意して、一行読み込むごとにカウントをした(28行目)。
そのあと、最後に先程推測した表示のコードを書いて(33行目)、
適する引数をメイン関数内で与えれば完成した。
実行結果は次のようになる。
\begin{lstlisting}[style = lstbash]
    $ ./kadai3 sample_100.txt
          10 sample_100.txt
\end{lstlisting}

だがこのままでは、
複数のファイルを読み込んだときにそれぞれの結果を表示するだけで
合計の行数を表示してくれないのでtotalありに改変する。
変数totalcntをグローバルに用意して、
関数が呼び出されるごとに1つのファイルの行数を数えるのでそれをtotalcntに加算した(34行目)。

最後にメイン関数内で、一つのファイルしか読み込まない場合はtotalを表示せず(45行目)、
複数のファイルを読み込む場合はtotalを表示するように分岐した(52行目)。
これでwc -lと同等の動きを実現できたので完成である。

複数のファイルを与えたときの実行結果は次のようになり正常に動いている。
\begin{lstlisting}[style = lstbash]
    $ ./kadai3 sample_10.txt sample_4096.txt
           1 sample_10.txt
         410 sample_4096.txt
         411 total
\end{lstlisting}

\clearpage
\subsection{レポート課題4}
\begin{shadebox}
    \fbox{演習\textbf{5}}で作成したプログラム(ソースコードex5-3.c)
    にリダイレクト「$>$」の機能を実装せよ。
    親プロセス側、小プロセス側どちらでファイルにリダイレクトしても構わないが、
    実行して正常に動作することを確かめること。
    また、実行結果をレポートに記載すること。
\end{shadebox}
\begin{lstlisting}[caption=ex5-3.c,label=ex5-3, style=lstC]
    #include <stdio.h>
    #include <stdlib.h>
    #include <unistd.h>
    #include <string.h>
    #include <sys/types.h>
    #include <sys/wait.h>
    
    #define MAX_LINE_IN 1000
    #define MAX_ARGS 30
    
    int main(int argc, char *argv[])
    {
        int pid, status;
        char line_in[MAX_LINE_IN];
        char *args[MAX_ARGS];
        int nargs;
    
        //tokenize
        for (;;)
        {
            printf("> ");
            if (fgets(line_in, MAX_LINE_IN, stdin) == NULL)
                exit(0);
            line_in[strlen(line_in) - 1] = '\0';
    
            char *token = strtok(line_in, " ");
            nargs = 0;
            args[nargs++] = token;
            while (token != NULL)
            {
                if (token != NULL)
                {
                    token = strtok(NULL, " ");
                    args[nargs++] = token;
                }
            }
            args[nargs] = '\0';
    
            if (strcmp(args[0], "exit") == 0)
            {
                return 0;
            }
    
            pid_t pid;
            pid = fork();
            if (pid)
            {
                int status;
                wait(&status);
            }
            else
            {
                execvp(args[0], args);
                printf("command not found\n");
                exit(1);
            }
        }
    
        return 0;
    }
\end{lstlisting}
\begin{lstlisting}[caption=kadai4.c,label=kadai4, style=lstC]
    #include <stdio.h>
    #include <stdlib.h>
    #include <unistd.h>
    #include <string.h>
    #include <sys/types.h>
    #include <sys/wait.h>
    #include <fcntl.h>
    
    #define MAX_LINE_IN 1000
    #define MAX_ARGS 30
    
    void redirect(char *args[], int potnt); // リダイレクトする関数
    
    int main(int argc, char *argv[])
    {
        int pid, status;
        char line_in[MAX_LINE_IN];
        char *args[MAX_ARGS];
        int nargs;
    
        //tokenize
        for (;;)
        {
            printf("> ");
            if (fgets(line_in, MAX_LINE_IN, stdin) == NULL)
                exit(0);
            line_in[strlen(line_in) - 1] = '\0';
    
            char *token = strtok(line_in, " ");
            nargs = 0;
            args[nargs++] = token;
            while (token != NULL)
            {
                if (token != NULL)
                {
                    token = strtok(NULL, " ");
                    args[nargs++] = token;
                }
            }
            args[nargs] = '\0';
    
            if (strcmp(args[0], "exit") == 0)
            {
                return 0;
            }
    
            if (fork() == 0)
            {
                int i;
                for (i = 0; args[i] != NULL; i++)
                {
                    if (strcmp(args[i], ">") == 0) // 出力先を切り替えるリダイレクトが存在すればそこに出力
                    {
                        redirect(args, i); // リダイレクトする関数
                    }
                }
                execvp(args[0], args);
                printf("command not found\n");
                exit(1);
            }
            else
            {
                wait(&status);
            }
        }
    
        return 0;
    }
    
    void redirect(char *args[], int point) // リダイレクトする関数
    {
        int fd;
    
        args[point] = '\0'; // ">"があった場所を終端文字にする
        fd = open(args[point + 1], O_WRONLY | O_CREAT, 0664);
        close(STDOUT_FILENO);
        dup(fd);
        execvp(args[0], args);
    }
\end{lstlisting}

\clearpage
上記に作成したソースコード(kadai4.c)を示した。
発想としては、argsの配列には半角スペースごとに分けられているので
その中で"$>$"があればリダイレクトを行うというコードを書こうと考えた。
よって、50行目と52行目を見れば分かるように、
全てのargsを"$>$"と比較することで"$>$"が存在すればリダイレクトの関数を呼び出している。

次に、リダイレクト関数の実装について(70から79行目)は、
もともと"$>$"があったargsの配列を終端文字にして(74行目)、
"$>$"があった次の配列に書かれているファイルをopenすることで実装している(75行目)。

実行結果は次のようになり、echoコマンドを用いたリダイレクトも
lsコマンドも実行できることが分かる。

\begin{lstlisting}[style = lstbash]
    $ ./kadai4
    > echo hello!! > test.txt
    > cat test.txt
    hello!!
    > ls
    kadai3   kadai3.c kadai4   kadai4.c test.txt
    > exit
\end{lstlisting}
\section{結論}

今回の実験を通してシステムコールを用いたシステムプログラミングを学び、
プロセスの基本概念や並列プログラミングの基礎を理解することができた。
また、実際にC言語を用いてそれらを実装かつ実行することができた。
さらに、シェルを自作して、
シェルの仕組みについて学べた。

%参考文献
\begin{thebibliography}{99}
    \label{sannkoubunnkenn_chapter}
    \bibitem[1]{rikadai}東京理科大学工学部情報工学科 情報工学実験1 2020年度
    東京理科大学工学部情報工学科出版
    \bibitem[2]{gets}C言語講座:gets( )とscanf( )の問題点の解決

    \url{http://www1.cts.ne.jp/clab/hsample/IO/IO16.html}

    最終閲覧日:2020/7/7

    \bibitem[3]{texcolor}LaTeXで色付きソースコードを貼り付け

    \url{http://yu00.hatenablog.com/entry/2015/05/14/214121}

    最終閲覧日:2020/7/7

\end{thebibliography}

\clearpage
\appendix
%%%%%%%%%%%%%%%%%%%%%%%%%%%%%%%%%%%%%%%%%%%%%%%%%%%%%%%
\end{document}
